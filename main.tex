\documentclass[letterpaper,11pt]{article}

\usepackage{latexsym}
\usepackage[empty]{fullpage}
\usepackage{titlesec}
\usepackage{marvosym}
\usepackage[usenames,dvipsnames]{color}
\usepackage{verbatim}
\usepackage{enumitem}

\usepackage[hidelinks]{hyperref}
\usepackage{fancyhdr}
\usepackage[english]{babel}
\usepackage{tabularx}
\input{glyphtounicode}

\pagestyle{fancy}
\fancyhf{} % clear all header and footer fields
\fancyfoot{}
\renewcommand{\headrulewidth}{0pt}
\renewcommand{\footrulewidth}{0pt}

% Adjust margins
\addtolength{\oddsidemargin}{-0.5in}
\addtolength{\evensidemargin}{-0.5in}
\addtolength{\textwidth}{1in}
\addtolength{\topmargin}{-.5in}
\addtolength{\textheight}{1.0in}

\urlstyle{same}

\raggedbottom
\raggedright
\setlength{\tabcolsep}{0in}

% Sections formatting
\titleformat{\section}{
  \vspace{-4pt}\scshape\raggedright\large
}{}{0em}{}[\color{black}\titlerule \vspace{-5pt}]

% Ensure that generate pdf is machine readable/ATS parsable
\pdfgentounicode=1

%-------------------------
% Custom commands
\newcommand{\resumeItem}[1]{
  \item\large{
    {#1 \vspace{0pt}}
  }
}

\newcommand{\resumeSubheading}[4]{
  \vspace{-2pt}\item
    \begin{tabular*}{0.97\textwidth}[t]{l@{\extracolsep{\fill}}r}
      \textbf{#1} & #2 \\
      \textit{\small#3} & \textit{\small #4} \\
    \end{tabular*}\vspace{-7pt}
}

\newcommand{\resumeSubSubheading}[2]{
    \item
    \begin{tabular*}{0.97\textwidth}{l@{\extracolsep{\fill}}r}
      \textit{\small#1} & \textit{\small #2} \\
    \end{tabular*}\vspace{-7pt}
}

\newcommand{\resumeProjectHeading}[2]{
    \item
    \begin{tabular*}{0.97\textwidth}{l@{\extracolsep{\fill}}r}
      \small#1 & #2 \\
    \end{tabular*}\vspace{-7pt}
}

\newcommand{\resumeSubItem}[1]{\resumeItem{#1}\vspace{-4pt}}

\renewcommand\labelitemii{$\vcenter{\hbox{\tiny$\bullet$}}$}

\newcommand{\resumeSubHeadingListStart}{\begin{itemize}[leftmargin=0.15in, label={}]}
\newcommand{\resumeSubHeadingListEnd}{\end{itemize}}
\newcommand{\resumeItemListStart}{\begin{itemize}}
\newcommand{\resumeItemListEnd}{\end{itemize}\vspace{-5pt}}

%%%%%%  RESUME STARTS HERE  %%%%%%%%%%%%%%%%%%%%%%%%%%%%

\begin{document}

%----------HEADING----------
\begin{center}
    \textbf{\Huge \scshape Koichi Kimura} \\ \vspace{1pt}
    \textbf{\Large \scshape Frontend React Developer} \\
    % 電話番号とアドレス。応募先が居住国外ならなら国際番号(例: +1)も入れるか電話番号は消してしまう
    \large 236 308-2236 $|$ Vancouver, Canada
\end{center}

\begin{center}
% Emailをhrefとunderline内に
    \href{mailto:Insert your email}{\underline{tech.kupumaru@gmail.com}} $|$ 
% LinkedInのリンク
    \href{https://www.linkedin.com/in/koichi-kimura-06ba14259/}{\underline{linkedin.com/in/koichi-kimura-06ba14259}} $|$
% Githubのリンク
    \href{https://github.com/kupuma-ru21}{\underline{https://github.com/kupuma-ru21}}
\end{center}

\section{About}
Software Engineer with 4 years of experience and a love of learning new skills.
My strong suits are Typescript, React, Next.js.
I've used Typescript and React for about four years, and Next.js for about two years.
I've already used App router by Next.js at work.
I'm strongly interested in maintainability of source codes and design patterns.

%-----------EXPERIENCE-----------
\section{Experience}
  \resumeSubHeadingListStart

  \resumeSubheading
    {Frontend Developer}{May 2024 - Now}
    {SORAJIMA}{Tokyo, Japan(Remote}
    \begin{itemize}
      \resumeItem{Selected and introduced a lib into a project.}
      \begin{itemize}
        \item {The project needed drag and drop feature so I decided to adopt dndkit. I considered react-dnd, react-beautiful-dnd and so on but most of them had some issues related to react 18 except dndkit. That's why I choose dndkit.}
      \end{itemize}
      \resumeItem{Improved developer experience.}
      \begin{itemize}
        \item {The project originally didn't have tsc check in CI. I added it and made the project more maintainable.}
      \end{itemize}
    \end{itemize}
    {Tech stack: react.js, next.js, chakra-ui, storybook, graphql, graphql-codegen, urql, react-hook-form, zod, esLint, prettier, pnpm, git, slack, jira, figma, golang}

  \resumeSubheading
      {Frontend and Backend Developer}{Feb 2023 - Dec 2023}
      {BUYSELL TECHNOLOGIES}{Tokyo, Japan(Remote}
      \begin{itemize}
        \resumeItem{Selected and introduced a lib into a project.}
        \begin{itemize}
          \item {The project originally used day.js, but I introduced date-fns because day.js caused a severe performance issue in a certain situation. Reasons to decided date-fns are that it was developed by Typescript and regularly maintained and installed by many developers compared to other date format libs.}
        \end{itemize}
        \resumeItem{Reviewed pull requests from all frontend developers.}
        \begin{itemize}
          \item {I was aware of making them avoid hasty abstractions like logical cohesion.}
        \end{itemize}
        \resumeItem{Created resolver of graphql with golang.}
        \begin{itemize}
          \item {To smoothly communicate with backend developers as a frontend developer, I was engaged in development of backend.}
        \end{itemize}
      \end{itemize}
      {Tech stack: react.js, next.js, storybook, graphql, graphql-codegen, apollo/client, material-ui, react-hook-form, zod, vitest, esLint, prettier, pnpm, git, slack, jira, figma, golang, sqlboiler}

    \resumeSubheading
      {Frontend Developer}{May 2021 - Feb 2023}
      {VISITS TECHNOLOGIES}{Tokyo, Japan(Remote}
      \begin{itemize}
        \resumeItem{Improved developer experiences by installing eslint prettier etc.}
        \begin{itemize}
          \item {I joined a project after the project just started, so there were no lint tools. I installed them and made the project more maintainable.}
        \end{itemize}
        \resumeItem{Reviewed pull requests from my coworkers.}
        \begin{itemize}
          \item {I was aware of making them to reduce time complexity and write semantic jsx.}
        \end{itemize}
        \resumeItem{Created test codes with jest and react-testing-library}
        \begin{itemize}
          \item {After I wrote application source codes, I wrote test source codes to improve quality of the project.}
        \end{itemize}
        \resumeItem{Updated React v17 to v18.}
        \begin{itemize}
          \item {It took me a week to update React v17 to v18. I managed to update many libraries and fix many bugs because of the updating.}
        \end{itemize}
      \end{itemize}
      {Tech stack: react.js, react-router, react-datepicker, react-dnd, react-i18next, react-testing-library, graphql, graphql-codegen, apollo/client, styled-components, react-hook-form, yup, react-i18next, jest, echarts, koa, esLint, prettier, yarn, git, slack, jira, figma}

    \resumeSubheading
      {Frontend Developer}{Sep 2019 - May 2021}
      {Gizmo inc.}{Tokyo, Japan}
      \begin{itemize}
        \resumeItem{Created UI and business logic for logistics web application used by SoftBank with Vue.js.}
        \resumeItem{Created UI and business logic for internet bank web application released by SONY with React.js.}
        \resumeItem{Technically supported my coworkers.}
      \end{itemize}
      {Tech Stack: react.js, redux, react-redux, reselect, styled-components, next.js, formik, yup, react-i18next, @adobe/target-react-component, axios, immer, ramda, redux-logger, redux-saga, typesafe-actions, dotenv-webpack, prism, jest, eslint, yarn, git, gitlab, aws, codecommit, sourcetree, slack, adobe xd, vue.js, vue router, vuex, vuetify, axios, joi, lodash, veeValidate, github}

  \resumeSubHeadingListEnd

 %-----------Education-----------
\section{Education}
 \begin{itemize}[leftmargin=0.15in, label={}]
    \small{\item{
     \textbf{Bachelor of engineering}{: Takushoku University, Tokyo, Japan (Apr 2015 - Mar 2019)} \\
     \textbf{Web development diploma}{: Cornerstone International College, Vancouver (May 2024 - Apr 2026)} \\
    }}
 \end{itemize}

 %-----------LANGUAGES-----------
\section{LANGUAGES}
 \begin{itemize}[leftmargin=0.15in, label={}]
    \small{\item{
     \textbf{English - Upper-intermediate} \\
     \textbf{Japanese - Native} \\
    }}
 \end{itemize}


%-------------------------------------------
\end{document}
